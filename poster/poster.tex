% Gemini theme
% https://github.com/anishathalye/gemini

\documentclass[final]{beamer}

% ====================
% Packages
% ====================

\usepackage[T1]{fontenc}
\usepackage{lmodern}
\usepackage[size=custom,width=88,height=91,scale=1.0]{beamerposter} 
% these dimensions are centimeters (checked PDF properties)
% we want to make a 4'x3' (121cmx91cm) poster with 1/2" margins on the height
%  then use a 121cmx88cm widthxheight poster
\usetheme{gemini}
\usecolortheme{gemini}
\usepackage{graphicx}
\usepackage{booktabs}
\usepackage{tikz}
\usepackage{pgfplots}
\pgfplotsset{compat=1.14}

% ====================
% Lengths
% ====================

% If you have N columns, choose \sepwidth and \colwidth such that
% (N+1)*\sepwidth + N*\colwidth = \paperwidth
\newlength{\sepwidth}
\newlength{\colwidth}
\setlength{\sepwidth}{0.025\paperwidth}
\setlength{\colwidth}{0.3\paperwidth}

\newcommand{\separatorcolumn}{\begin{column}{\sepwidth}\end{column}}

% ====================
% Title
% ====================

\title{Addressing Backgrounds for CUPID on Three Fronts}

\author{Joseph Camilleri} %\inst{1} \and Ben Bitdiddle \inst{2} \and Lem E. Tweakit \inst{2}}

\institute[shortinst]{Virginia Tech} %\samelineand \inst{2} Another Institute}

% ====================
% Footer (optional)
% ====================

\footercontent{
  % \href{https://www.example.com}{https://www.example.com} \hfill
  Center for Neutrino Phyiscs Research Day 2022, Blacksburg, VA  \hfill
  \href{mailto:jcamilleri@vt.edu}{jcamilleri@vt.edu}}
% (can be left out to remove footer)

% ====================
% Logo (optional)
% ====================

% use this to include logos on the left and/or right side of the header:
\logoright{\includegraphics[height=7cm]{~/research/figures/CNPday_poster/vtlogo.png}}
\logoleft{\includegraphics[height=7cm]{~/research/figures/CNPday_poster/logo_cupid.png}}

% ====================
% Body
% ====================

\begin{document}

\begin{frame}[t]
\begin{columns}[t]
\separatorcolumn

\begin{column}{\colwidth}

  \begin{block}{Neutrinoless Double Beta Decay}
    
      \begin{eqnarray*}
      \hat{\nu} = \int \frac{d^3 \overline{k}}{(2\pi)^3 2 \omega_k} \sum_s
      \bigg[\ \textcolor{blue}{\hat{b}(\overline{k},s)}u(\overline{k},s)e^{-ikx}
       + \ \textcolor{red}{\hat{b}^{\dagger}(\overline{k},s)}v(\overline{k},s)e^{+ikx} \bigg]
      \end{eqnarray*}
      
      \begin{eqnarray*}
      \mathcal{L}_{cc} = -\frac{g}{\sqrt{2}}\bigg[\textcolor{blue}{\overline{e}\gamma^{\mu}\left(\frac{1-\gamma_5}{2}\right)\nu W^{-}_{\mu}} 
      + \textcolor{red}{\overline{\nu}\gamma^{\mu}\left(\frac{1-\gamma_5}{2}\right)e W^{+}_{\mu}}\bigg]
      \end{eqnarray*}
      
      \begin{columns}[c] % The "c" option specifies centered vertical alignment while the "t" option is used for top vertical alignment
        
        \column{.4\colwidth} % Left column and width
        \textcolor{blue} {obeys lepton number conservation} 
        
        \column{.4\colwidth} % Right column and width
        \textcolor{red} {allowed for majorana fermions, but relativistically-suppressed}
        
      \end{columns}
      
      \vspace{1cm}
    \begin{columns}[c] % The "c" option specifies centered vertical alignment while the "t" option is used for top vertical alignment
        
        \column{.35\colwidth} % Left column and width
        \includegraphics[width=.35\colwidth]{~/research/figures/prelim/2vbb_feynman.png}
        {\footnotesize Two-neutrino double beta decay $T_{1/2}^{2\nu\beta\beta} \sim 10^{19}-10^{24} yr$}
        
        \column{.35\colwidth} % Right column and width
        \includegraphics[width=.35\colwidth]{~/research/figures/prelim/0vbb_feynman.png}
        {\footnotesize Neutrino-less double beta decay $T_{1/2}^{0\nu\beta\beta} > 10^{26} yr$}
        
    \end{columns}
  \end{block}

  \begin{block}{CUPID experiment}
    \vspace{2cm}
    \begin{columns}[c] % The "c" option specifies centered vertical alignment while the "t" option is used for top vertical alignment
      
      \column{.4\colwidth} % Left column and width
      \includegraphics[width=.4\colwidth]{~/research/figures/CNPday_poster/cryostat.png}
      % {\footnotesize Rendering of proposed CUPID array of Li$_2$MoO$_4$ crystals}
      
      \column{.5\colwidth} % Right column and width
      \includegraphics[width=.5\colwidth]{~/research/figures/DNP2021/scintillation.png}
      % {\footnotesize Rendering of proposed CUPID array of Li$_2$MoO$_4$ crystals}
      
    \end{columns}

    \begin{itemize}
      % \setlength\itemsep{2em}
      \item Proposed $0\nu\beta\beta$ search using bolometric array of 1596 Li$_2$MoO$_4$ crystals, to be deployed in the CUORE \ cryostat\footnotemark .
      \item Aims to eliminate dominant background of alpha particles present in CUORE via thermal + scintillation signals.
      \item \textbf{Are new backgrounds introduced with using a new isotope for the bolometers?}
    \end{itemize}


    \begin{columns}[c] % The "c" option specifies centered vertical alignment while the "t" option is used for top vertical alignment
      
      \column{.4\colwidth} % Left column and width
        \begin{eqnarray*}
        F_{0\nu\beta\beta} \propto  \left[ a\epsilon \sqrt{\frac{Mt}{\textcolor{red}{b}\Delta}}\right]
        \end{eqnarray*}      
      
      \column{.4\colwidth} % Right column and width
        \begin{center}
        \resizebox{10cm}{!}{         
        \begin{tabular}{||c c||} 
         \hline
         M & source mass (kg)\\  
         \hline
         b & \textcolor{red}{bkg index} $counts/(kg\cdot yr\cdot keV)$ \\ 
         \hline
         a & isotopic abundance ($\%$) \\
         \hline
         t & exposure (yr)  \\
         \hline
         $\Delta$ & energy resolution (keV)\\
         \hline
         $\epsilon$ & detection efficiency \\
         \hline
        \end{tabular}}
        \end{center}
      
    \end{columns}


  \end{block}

  \begin{block}{Monte-Carlo simulations of CUPID array}

    CUPID's irreducible background is $2\nu\beta\beta$. This is an observed, but extrememly rare process like $0\nu\beta\beta$
    \vspace{2cm}
    \begin{columns}[c] % The "c" option specifies centered vertical alignment while the "t" option is used for top vertical alignment
      
      \column{.5\colwidth} % Left column and width
      \includegraphics[width=.45\colwidth]{~/research/figures/DNP2021/gamma2.jpg}
      \includegraphics[width=.45\colwidth]{~/research/figures/DNP2021/cupid_array.png}
      % {\footnotesize Rendering of proposed CUPID array of Li$_2$MoO$_4$ crystals}
      
      \column{.5\colwidth} % Right column and width
      \includegraphics[width=.45\colwidth]{~/research/figures/DNP2021/u238.jpg}
      % {\footnotesize Rendering of proposed CUPID array of Li$_2$MoO$_4$ crystals}
      
    \end{columns}


  \end{block}

  

\end{column}

\separatorcolumn

\begin{column}{\colwidth}
      
      \begin{eqnarray*}
      ^{238}U \xrightarrow{\alpha} \  ^{234}Th \xrightarrow \  ... \\ ^{214}Bi \xrightarrow{\beta} \  ^{214}Po \xrightarrow{\alpha} \ ^{210}Pb \ ...
      \end{eqnarray*}
      $T_{1/2} \ ^{214}$Po $\sim$ 160$\mu$s \ \ $T_{1/2} \ ^{214}$Bi $\sim$ 20 minutes
    \includegraphics[width=0.8\colwidth]{~/research/figures/DNP2021/2vbb_2.jpg}
  
  \begin{block}{Neutron activation at DUKE: TUNL TANDEM beam}

  \vspace{2cm}
  \begin{columns}[c] % The "c" option specifies centered vertical alignment while the "t" option is used for top vertical alignment
      
      \column{.4\colwidth} % Left column and width
      \includegraphics[width=.4\colwidth]{~/research/figures/CNPday_poster/TUNL.png}
      {\footnotesize TUNL's neutron beam apparatus}    
      
      \column{.5\colwidth} % Right column and width
      \includegraphics[width=.5\colwidth]{~/research/figures/CNPday_poster/8MeV_TOF.png}
      {\footnotesize Time of flight spectrum of events from TANDEM accelerator detected in HPGe} 
  \end{columns}


   \includegraphics[width=0.8\colwidth]{~/research/figures/CNPday_poster/nmontof.png}
      {\footnotesize PSD vs TOF to validate our TOF cut performed on data}


 
  \end{block}

  

  

\end{column}

\separatorcolumn

\begin{column}{\colwidth}
      \vspace{5cm}
      \includegraphics[height=20cm]{~/research/figures/CNPday_poster/tunlphoto.jpg}
      {\footnotesize From left to right: Daniel Mayer (MIT), Werner Tornow(Duke), Erin Hansen(Berkeley), Joe Camilleri (VT), Sean Finch (Duke)}
      \includegraphics[width=0.9\colwidth]{~/research/figures/CNPday_poster/twoplus.png}
      {\footnotesize Neutron spectrum and first excited state of $^{100}Mo$}


  

  \begin{block}{References}

    \nocite{*}
    \footnotesize{\bibliographystyle{plain}\bibliography{poster}}

  \end{block}

\end{column}

\separatorcolumn
\end{columns}
\end{frame}

\end{document}
