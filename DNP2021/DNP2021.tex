%%%%%%%%%%%%%%%%%%%%%%%%%%%%%%%%%%%%%%%%%
% Beamer Presentation
% LaTeX Template
% Version 1.0 (10/11/12)
%
% This template has been downloaded from:
% http://www.LaTeXTemplates.com
%
% License:
% CC BY-NC-SA 3.0 (http://creativecommons.org/licenses/by-nc-sa/3.0/)
%
%%%%%%%%%%%%%%%%%%%%%%%%%%%%%%%%%%%%%%%%%

%----------------------------------------------------------------------------------------
%	PACKAGES AND THEMES
%----------------------------------------------------------------------------------------

\documentclass{beamer}

\mode<presentation> {
	
	% The Beamer class comes with a number of default slide themes
	% which change the colors and layouts of slides. Below this is a list
	% of all the themes, uncomment each in turn to see what they look like.
	
	%\usetheme{default}
	%\usetheme{AnnArbor}
	%\usetheme{Antibes}
	%\usetheme{Bergen}
	%\usetheme{Berkeley}
	%\usetheme{Berlin}
	%\usetheme{Boadilla}
	%\usetheme{CambridgeUS}
	%\usetheme{Copenhagen}
	%\usetheme{Darmstadt}
	%\usetheme{Dresden}
	%\usetheme{Frankfurt}
	%\usetheme{Goettingen}
	\usetheme{Hannover}
	%\usetheme{Ilmenau}
	%\usetheme{JuanLesPins}
	%\usetheme{Luebeck}
	%\usetheme{Madrid}
	%\usetheme{Malmoe}
	%\usetheme{Marburg}
	%\usetheme{Montpellier}
	%\usetheme{PaloAlto}
	%\usetheme{Pittsburgh}
	%\usetheme{Rochester}
	%\usetheme{Singapore}
	%\usetheme{Szeged}
	%\usetheme{Warsaw}
	
	% As well as themes, the Beamer class has a number of color themes
	% for any slide theme. Uncomment each of these in turn to see how it
	% changes the colors of your current slide theme.
	
	%\usecolortheme{albatross}
	%\usecolortheme{beaver}
	%\usecolortheme{beetle}
	%\usecolortheme{crane}
	%\usecolortheme{dolphin}
	%\usecolortheme{dove}
	%\usecolortheme{fly}
	%\usecolortheme{lily}
	%\usecolortheme{orchid}
	%\usecolortheme{rose}
	%\usecolortheme{seagull}
	\usecolortheme{seahorse}
	%\usecolortheme{whale}
	%\usecolortheme{wolverine}
	
	%\setbeamertemplate{footline} % To remove the footer line in all slides uncomment this line
	\setbeamertemplate{footline}[page number] % To replace the footer line in all slides with a simple slide count uncomment this line
	
	\setbeamertemplate{navigation symbols}{} % To remove the navigation symbols from the bottom of all slides uncomment this line
	\setbeamertemplate{itemize/enumerate body begin}{\scriptsize} % this controls the size of the text in the enumerated lists
}
%\usepackage{enumitem} % for formatting itemized lists (/enumerate for example is used a lot here)
\usepackage{graphicx} % Allows including images
\usepackage{booktabs} % Allows the use of \toprule, \midrule and \bottomrule in tables
\usepackage[backend=biber, style=authoryear, bibencoding=utf8]{biblatex}
\addbibresource{~/research/tex/bib2.bib}


%----------------------------------------------------------------------------------------
%	TITLE PAGE
%----------------------------------------------------------------------------------------

\title[CUPID array]{Optimization of integration time and distance cut in the CUPID array} % The short title appears at the bottom of every slide, the full title is only on the title page

\author{Joe Camilleri} % Your name
\institute[Virginia Tech] % Your institution as it will appear on the bottom of every slide, may be shorthand to save space
{
	Virginia Tech \\ % Your institution for the title page
	\medskip
	DNP October 2021 \\
	\medskip
%	\textit{jcamilleri@vt.edu} % Your email address
	\textbf{Mini-Symposium: Neutrinos and Nuclei XII: Double Beta Decay Analysis Techniques}
}

\date{} % Date, can be changed to a custom date

\begin{document}

	\begin{frame}
		\titlepage % Print the title page as the first slide
	\end{frame}
	
%	\begin{frame}
%		\frametitle{Table of contents} % Table of contents slide, comment this block out to remove it
%		%\tableofcontents % Throughout your presentation, if you choose to use \section{} and \subsection{} commands, these will automatically be printed on this slide as an overview of your presentation
%	\end{frame}
	
	
	%------------------------------------------------
%	\section{CUPID}
	% ----------------------------------------------
	
%	\begin{frame}
%		\frametitle{CUPID experiment}
%		\begin{columns}[c] % The "c" option specifies centered vertical alignment while the "t" option is used for top vertical alignment
%			
%			\column{.45\textwidth} % Left column and width
%			%\textbf{Heading}
%			\begin{itemize}
%				\setlength\itemsep{2em}
%				\item Proposed $0\nu\beta\beta$ search using bolometric array of 1596 Li$_2$MoO$_4$ crystals, deployed in the CUORE\footnotemark cryostat.
%				\item Aims to eliminate dominant background of alpha particles present in CUORE.
%				\item \textbf{Are new backgrounds introduced with a using a new element for the bolometers?}
%			\end{itemize}
%			
%			\column{.5\textwidth} % Right column and width
%			\includegraphics[width=\linewidth]{~/research/figures/DNP2021/cupid_array.png}
%			{\footnotesize Rendering of proposed CUPID array of Li$_2$MoO$_4$ crystals}
%			
%		\end{columns}
%		\footcitetext{squid}
%	\end{frame}
	
	\begin{frame}
		\frametitle{CUPID experiment}
		\begin{columns}[c] % The "c" option specifies centered vertical alignment while the "t" option is used for top vertical alignment
			
			\column{.45\textwidth} % Left column and width
			%\textbf{Heading}
			\begin{itemize}
				\setlength\itemsep{2em}
				\item Proposed $0\nu\beta\beta$ search using bolometric array of 1596 Li$_2$MoO$_4$ crystals, deployed in the CUORE cryostat.
				\item Aims to eliminate dominant background of alpha particles present in CUORE.
				\item \textbf{Are new backgrounds introduced with using a new element for the bolometers?}
			\end{itemize}
			
			\column{.5\textwidth} % Right column and width
			\includegraphics[width=\linewidth]{~/research/figures/DNP2021/cupid_array.png}
			{\footnotesize Rendering of proposed CUPID array of Li$_2$MoO$_4$ crystals}
			
		\end{columns}
%		\footcitetext{squid}
	\end{frame}
	
	%------------------------------------------------

	\begin{frame}
		\frametitle{Lithium molybdate}
		\begin{columns}[c] % The "c" option specifies centered vertical alignment while the "t" option is used for top vertical alignment
			
			\column{.45\textwidth} % Left column and width
			%\textbf{Heading}
			\begin{itemize}
				\setlength\itemsep{2em}
				\item Li$_2$MoO$_4$ crystals allow for discrimination of $\alpha$ backgrounds from $\beta\beta$ events (Q=3034keV) via thermal + scintillation signals.
				\item relatively high isotopic abundance of $^{100}$Mo (10\%)
				\item enrichment above 95\% already demonstrated in CUPID-Mo \cite{}
			\end{itemize}
			
			\column{.5\textwidth} % Right column and width
			\includegraphics[width=\linewidth]{~/research/figures/DNP2021/scintillation.png}
		\end{columns}
	\end{frame}

	%------------------------------------------------	
	
	%------------------------------------------------

	\begin{frame}
		\frametitle{2$\nu\beta\beta$ events and muons}
		\begin{columns}[c] % The "c" option specifies centered vertical alignment while the "t" option is used for top vertical alignment
			
			\column{.45\textwidth} % Left column and width
			%\textbf{Heading}
			\begin{itemize}
				\setlength\itemsep{2em}				
				\item The rate of $2\nu\beta\beta$ events is not negligible in CUPID array
				\item Minimizing the distance cut helps avoid mis-labelling random $2\nu\beta\beta$ coincidences as multiplicity 2.
				\item Assuming a simple muon veto geometry, increasing the distance cut rejects more muon events.
			\end{itemize}
			
			\column{.5\textwidth} % Right column and width
			\begin{eqnarray*}
			T_{1/2} = 7.1* 10^{18} yr \\ 
			\rightarrow \text{rate} \sim 3mHz
			\end{eqnarray*}			
			
			\begin{figure}
			\includegraphics[width=\linewidth]{~/research/figures/DNP2021/muon_veto.png}
			%\caption{gnarly}
			\end{figure}
			
		\end{columns}
%		\footcitetext{chernyak}
	\end{frame}

	%------------------------------------------------

	%------------------------------------------------

	\begin{frame}
		\frametitle{Distance cut in the CUPID array}
		\begin{columns}[c] % The "c" option specifies centered vertical alignment while the "t" option is used for top vertical alignment
			
			\column{.45\textwidth} % Left column and width
			%\textbf{Heading}
			\begin{itemize}
				\setlength\itemsep{2em}
				\item monte-carlo simulation of 1 million 2.6MeV gamma rays in the crystal volume.
				\item With this energy, we expect multiple scattering events in the crystals (cite scattering length)
				\item $2>$ multiplicity events are discarded
			\end{itemize}
			
			\column{.5\textwidth} % Right column and width
			
			\includegraphics[width=\linewidth]{~/research/figures/DNP2021/gamma2.jpg}
%			\caption{gnarly}
			
			
		\end{columns}
	\end{frame}

	%------------------------------------------------

	%------------------------------------------------

	\begin{frame}
		\frametitle{Integration time in the CUPID array}
		\begin{columns}[c] % The "c" option specifies centered vertical alignment while the "t" option is used for top vertical alignment
			
			\column{.45\textwidth} % Left column and width
			%\textbf{Heading}
			\begin{eqnarray*}
			^{238}U \xrightarrow{\alpha} \  ^{234}Th \xrightarrow \  ... \\ ^{214}Bi \xrightarrow{\beta} \  ^{214}Po \xrightarrow{\alpha} \ ^{210}Pb \ ...
			\end{eqnarray*}
			\begin{itemize}
				\setlength\itemsep{2em}
				\item 100,000 uranium-238 events (full chain)
				\item $T_{1/2} \ ^{214}$Po $\sim$ 160$\mu$s
				\item $T_{1/2} \ ^{214}$Bi $\sim$ 20 minutes
			\end{itemize}
			
			\column{.5\textwidth} % Right column and width
%			\begin{figure}
			\hspace*{0.3cm}\includegraphics[width=5cm, scale=0.20]{~/research/figures/DNP2021/u238.jpg}
			\medskip\medskip
			\hspace*{0.3cm}\includegraphics[width=5cm, scale=0.20]{~/research/figures/DNP2021/u238_2.jpg}
%			\caption{gnarly}
%			\end{figure}
			
		\end{columns}
	\end{frame}

	%------------------------------------------------

%------------------------------------------------

	\begin{frame}
		\frametitle{2$\nu\beta\beta$ efficiency simulation}
		\begin{columns}[c] % The "c" option specifies centered vertical alignment while the "t" option is used for top vertical alignment
			
			\column{.45\textwidth} % Left column and width
			%\textbf{Heading}
			\begin{itemize}
				\setlength\itemsep{2em}
				\item sensitivity studies expect on the order of 90$\%$ efficiency for 2$\nu\beta\beta$ efficiency
				\item larger distance cut causes more random coincidences and more variation with integration time
				\item bremsstrahlung, escape, random coincidences are primary contributors
			\end{itemize}
			
			\column{.5\textwidth} % Right column and width
%			\begin{figure}
			\hspace*{0.5cm}\includegraphics[width=5cm, scale=0.5]{~/research/figures/DNP2021/2vbb_3.jpg}
			\medskip
			\hspace*{0.5cm}\includegraphics[width=5cm, scale=0.5]{~/research/figures/DNP2021/2vbb_2.jpg}
%			\caption{gnarly}
%			\end{figure}
			
		\end{columns}
	\end{frame}

	%------------------------------------------------

%------------------------------------------------

	\begin{frame}
		\frametitle{future work and the muon background}
		\begin{columns}[c] % The "c" option specifies centered vertical alignment while the "t" option is used for top vertical alignment
			
			\column{.45\textwidth} % Left column and width
			%\textbf{Heading}
			\begin{itemize}
				\setlength\itemsep{2em}
				\item incorporate scintillating muon veto geometry into monte-carlo simulations
				\item muon flux at LNGS is $3\cdot 10^{-8} \ \text{muons} \ / \left(s\cdot cm^2\right)$
				\item muon suppression versus distance cut optimization.
			\end{itemize}
			
			\column{.5\textwidth} % Right column and width
			\includegraphics[scale=0.2]{~/research/figures/DNP2021/cupid_w_veto.jpeg}

			
		\end{columns}
	\end{frame}

	%------------------------------------------------

	%------------------------------------------------

%	\begin{frame}
%		\frametitle{test frame for copying}
%		\begin{columns}[c] % The "c" option specifies centered vertical alignment while the "t" option is used for top vertical alignment
%			
%			\column{.45\textwidth} % Left column and width
%			%\textbf{Heading}
%			\begin{itemize}
%				\item guy
%				\item man
%				\item dude
%			\end{itemize}
%			
%			\column{.5\textwidth} % Right column and width
%			\begin{figure}
%			\includegraphics[width=\linewidth]{~/research/figures/DNP/cupid_array.png}
%			\caption{gnarly}
%			\end{figure}
%			
%		\end{columns}
%	\end{frame}
%
%	%------------------------------------------------
%
%	\begin{frame}
%		\frametitle{CUORE and alpha particles}
%			
%		CUORE is a massive bolometric detector searching for $0\nu\beta\beta$ decay in $^{130}$Te.
%	
%	\end{frame}
%	%----------------------------------------------------------------------------------------
%	%	PRESENTATION SLIDES
%	%----------------------------------------------------------------------------------------
%	
%	%------------------------------------------------
%%	\section{First Section} % Sections can be created in order to organize your presentation into discrete blocks, all sections and subsections are automatically printed in the table of contents as an overview of the talk
%	%------------------------------------------------
%	
%%	\subsection{Subsection Example} % A subsection can be created just before a set of slides with a common theme to further break down your presentation into chunks
%	
%	
%	
%	%------------------------------------------------
%	
%
%
%	%------------------------------------------------
%
%
%	
%	%------------------------------------------------
%	
%	\begin{frame}
%		\frametitle{Blocks of Highlighted Text}
%		\begin{block}{Block 1}
%			Lorem ipsum dolor sit amet, consectetur adipiscing elit. Integer lectus nisl, ultricies in feugiat rutrum, porttitor sit amet augue. Aliquam ut tortor mauris. Sed volutpat ante purus, quis accumsan dolor.
%		\end{block}
%		
%		\begin{block}{Block 2}
%			Pellentesque sed tellus purus. Class aptent taciti sociosqu ad litora torquent per conubia nostra, per inceptos himenaeos. Vestibulum quis magna at risus dictum tempor eu vitae velit.
%		\end{block}
%		
%		\begin{block}{Block 3}
%			Suspendisse tincidunt sagittis gravida. Curabitur condimentum, enim sed venenatis rutrum, ipsum neque consectetur orci, sed blandit justo nisi ac lacus.
%		\end{block}
%	\end{frame}
%	
%	%------------------------------------------------
%	
%	\begin{frame}
%		\frametitle{Multiple Columns}
%		\begin{columns}[c] % The "c" option specifies centered vertical alignment while the "t" option is used for top vertical alignment
%			
%			\column{.45\textwidth} % Left column and width
%			\textbf{Heading}
%			\begin{enumerate}
%				\item Statement
%				\item Explanation
%				\item Example
%			\end{enumerate}
%			
%			\column{.5\textwidth} % Right column and width
%			Lorem ipsum dolor sit amet, consectetur adipiscing elit. Integer lectus nisl, ultricies in feugiat rutrum, porttitor sit amet augue. Aliquam ut tortor mauris. Sed volutpat ante purus, quis accumsan dolor.
%			
%		\end{columns}
%	\end{frame}
%	
%	%------------------------------------------------
%	%\section{Second Section}
%	%------------------------------------------------
%	
%	\begin{frame}
%		\frametitle{Table}
%		\begin{table}
%			\begin{tabular}{l l l}
%				\toprule
%				\textbf{Treatments} & \textbf{Response 1} & \textbf{Response 2}\\
%				\midrule
%				Treatment 1 & 0.0003262 & 0.562 \\
%				Treatment 2 & 0.0015681 & 0.910 \\
%				Treatment 3 & 0.0009271 & 0.296 \\
%				\bottomrule
%			\end{tabular}
%			\caption{Table caption}
%		\end{table}
%	\end{frame}
%	
%	%------------------------------------------------
%	
%	\begin{frame}
%		\frametitle{Theorem}
%		\begin{theorem}[Mass--energy equivalence]
%			$E = mc^2$
%		\end{theorem}
%	\end{frame}
%	
%	%------------------------------------------------
%	
%	\begin{frame}[fragile] % Need to use the fragile option when verbatim is used in the slide
%		\frametitle{Verbatim}
%		\begin{example}[Theorem Slide Code]
%			\begin{verbatim}
%				\begin{frame}
%					\frametitle{Theorem}
%					\begin{theorem}[Mass--energy equivalence]
%						$E = mc^2$
%					\end{theorem}
%			\end{frame}\end{verbatim}
%		\end{example}
%	\end{frame}
%	
%	%------------------------------------------------
%	
%	\begin{frame}
%		\frametitle{Figure}
%		Uncomment the code on this slide to include your own image from the same directory as the template .TeX file.
%		%\begin{figure}
%		%\includegraphics[width=0.8\linewidth]{test}
%		%\end{figure}
%	\end{frame}
%	
%	%------------------------------------------------
%	
%	\begin{frame}[fragile] % Need to use the fragile option when verbatim is used in the slide
%		\frametitle{Citation}
%		An example of the \verb|\cite| command to cite within the presentation:\\~
%		
%		This statement requires citation \cite{p1}.
%	\end{frame}
%	
%	%------------------------------------------------
%	
%	\begin{frame}
%		\frametitle{References}
%		\footnotesize{
%			\begin{thebibliography}{99} % Beamer does not support BibTeX so references must be inserted manually as below
%				\bibitem[Smith, 2012]{p1} John Smith (2012)
%				\newblock Title of the publication
%				\newblock \emph{Journal Name} 12(3), 45 -- 678.
%			\end{thebibliography}
%		}
%	\end{frame}
%	
%	%------------------------------------------------
%	
%	\begin{frame}
%		\Huge{\centerline{The End}}
%	\end{frame}
	
	%----------------------------------------------------------------------------------------
	
\end{document}