%%%%%%%%%%%%%%%%%%%%%%%%%%%%%%%%%%%%%%%%%
% Beamer Presentation
% LaTeX Template
% Version 1.0 (10/11/12)
%
% This template has been downloaded from:
% http://www.LaTeXTemplates.com
%
% License:
% CC BY-NC-SA 3.0 (http://creativecommons.org/licenses/by-nc-sa/3.0/)
%
%%%%%%%%%%%%%%%%%%%%%%%%%%%%%%%%%%%%%%%%%

%----------------------------------------------------------------------------------------
%	PACKAGES AND THEMES
%----------------------------------------------------------------------------------------

\documentclass{beamer}
\usepackage{filecontents}

\begin{filecontents}{mybib.bib}
@article{cryostat,
   title={The CUORE cryostat: An infrastructure for rare event searches at millikelvin temperatures},
   volume={102},
   ISSN={0011-2275},
   url={http://dx.doi.org/10.1016/j.cryogenics.2019.06.011},
   DOI={10.1016/j.cryogenics.2019.06.011},
   journal={Cryogenics},
   publisher={Elsevier BV},
   author={Alduino, C. and Alessandria, F. and Balata, M. and Biare, D. and Biassoni, M. and Bucci, C. and Caminata, A. and Canonica, L. and Cappelli, L. and Ceruti, G. and et al.},
   year={2019},
   month={Sep},
   pages={9–21}
}
\end{filecontents}
\mode<presentation> {
	
	% The Beamer class comes with a number of default slide themes
	% which change the colors and layouts of slides. Below this is a list
	% of all the themes, uncomment each in turn to see what they look like.
	
	%\usetheme{default}
	%\usetheme{AnnArbor}
	%\usetheme{Antibes}
	%\usetheme{Bergen}
	%\usetheme{Berkeley}
	\usetheme{Berlin}
	%\usetheme{Boadilla}
	%\usetheme{CambridgeUS}
	%\usetheme{Copenhagen}
	%\usetheme{Darmstadt}
	%\usetheme{Dresden}
	%\usetheme{Frankfurt}
	%\usetheme{Goettingen}
	%\usetheme{Hannover}
	%\usetheme{Ilmenau}
	%\usetheme{JuanLesPins}
	%\usetheme{Luebeck}
	%\usetheme{Madrid}
	%\usetheme{Malmoe}
	%\usetheme{Marburg}
	%\usetheme{Montpellier}
	%\usetheme{PaloAlto}
	%\usetheme{Pittsburgh}
	%\usetheme{Rochester}
	%\usetheme{Singapore}
	%\usetheme{Szeged}
	%\usetheme{Warsaw}
	
	% As well as themes, the Beamer class has a number of color themes
	% for any slide theme. Uncomment each of these in turn to see how it
	% changes the colors of your current slide theme.
	
	%\usecolortheme{albatross}
	%\usecolortheme{beaver}
	%\usecolortheme{beetle}
	%\usecolortheme{crane}
	%\usecolortheme{dolphin}
	%\usecolortheme{dove}
	%\usecolortheme{fly}
	%\usecolortheme{lily}
	%\usecolortheme{orchid}
	%\usecolortheme{rose}
	%\usecolortheme{seagull}
	\usecolortheme{seahorse}
	%\usecolortheme{whale}
	%\usecolortheme{wolverine}
	
	%\setbeamertemplate{footline} % To remove the footer line in all slides uncomment this line
	\setbeamertemplate{footline}[page number] % To replace the footer line in all slides with a simple slide count uncomment this line
	
	\setbeamertemplate{navigation symbols}{} % To remove the navigation symbols from the bottom of all slides uncomment this line
	\setbeamertemplate{itemize/enumerate body begin}{\scriptsize} % this controls the size of the text in the enumerated lists
}
%\usepackage{enumitem} % for formatting itemized lists (/enumerate for example is used a lot here)
\usepackage{graphicx} % Allows including images
\usepackage{booktabs} % Allows the use of \toprule, \midrule and \bottomrule in tables
\usepackage[backend=biber, style=authoryear, url=true, bibencoding=utf8]{biblatex}
%\addbibresource{~/research/tex/bib2.bib}
\bibliography{mybib}
\usepackage{cancel} % for the slash symbol found through \cancel{} command
\usepackage{tikz} % for circling stuff in mathmode
\usepackage{mathtools}% also for circling stuff in mathmode.
\usepackage{bm} % for \boldsymbol{}
\makeatletter
\newcommand\mathcircled[1]{%
  \mathpalette\@mathcircled{#1}%
}
\newcommand\@mathcircled[2]{%
  \tikz[baseline=(math.base)] \node[draw,circle,inner sep=1pt, color=red] (math) {$\m@th#1#2$};%
}
\makeatother

%----------------------------------------------------------------------------------------
%	TITLE PAGE
%----------------------------------------------------------------------------------------

\title[CUPID array]{Preliminary Examination} % The short title appears at the bottom of every slide, the full title is only on the title page

\author{Joe Camilleri} % Your name
\institute[Virginia Tech] % Your institution as it will appear on the bottom of every slide, may be shorthand to save space
{
	Virginia Tech \\ % Your institution for the title page
	\medskip
	\medskip
%	\textit{jcamilleri@vt.edu} % Your email address
	\textbf{prelim exam}
}

\date{} % Date, can be changed to a custom date

\begin{document}

	\begin{frame}
		\titlepage % Print the title page as the first slide
	\end{frame}
	
%	\begin{frame}
%		\frametitle{Table of contents} % Table of contents slide, comment this block out to remove it
%		%\tableofcontents % Throughout your presentation, if you choose to use \section{} and \subsection{} commands, these will automatically be printed on this slide as an overview of your presentation
%	\end{frame}
	
	
	%------------------------------------------------
\section{Motivation and Background}
	% ----------------------------------------------
	
	\begin{frame}
		\frametitle{Dirac equation and charge conjugation}
		\begin{columns}[c] % The "c" option specifies centered vertical alignment while the "t" option is used for top vertical alignment
			
			\column{.45\textwidth} % Left column and width
			%\textbf{Heading}
			\begin{eqnarray*}
			\mathcal{L} = \frac{i}{2} \big[ \ \overline{\psi}\gamma^{\mu}\left(\partial_{\mu}\psi\right) \\
			- \left(\partial_{\mu}\overline{\psi}\right)\gamma^{\mu}\psi\big] - \mathcircled{ m\overline{\psi}\psi}
			\end{eqnarray*}						
			
				\begin{eqnarray*}
				\boldsymbol{e^- \Longleftrightarrow \ e^+} \\
				\hat{\psi} \rightarrow \hat{\psi}^C = C\gamma ^0 \hat{\psi}^*\\
				C \ = \ 
				\begin{pmatrix}
				i\sigma_2 & 0 \\
				0 & -i\sigma_2				
				\end{pmatrix}
				\end{eqnarray*} 
			
			\column{.5\textwidth} % Right column and width
			\begin{eqnarray*}
			\left(i\cancel{\partial} - m\right)\psi = 0 \\
			\psi = \begin{pmatrix}
			 \varphi \\ 
			 \chi 
			 \end{pmatrix}
			\end{eqnarray*}
			\includegraphics[width=\textwidth]{~/research/figures/prelim/dirac1928.png}
			
		\end{columns}
	\end{frame}
	
	\begin{frame}
		\frametitle{Majorana fermions and neutrinos}
		\begin{columns}[c] % The "c" option specifies centered vertical alignment while the "t" option is used for top vertical alignment
			
			\column{.45\textwidth} % Left column and width
			%\textbf{Heading}					
			\begin{itemize}
			\item Majorana fermions can be seen as solutions to the Dirac equation with an extra constraint
			\item $\psi = \psi^C$
			\item implies these fermions are \textbf{chargeless}, and their \textbf{own anti-particles}
			\end{itemize}
			
			\column{.5\textwidth} % Right column and width
			{\footnotesize The left and right handed components of the spinor are no longer independent}
			\begin{eqnarray*}
			\Psi_{majorana} = \psi_L + \psi_R \\
			\end{eqnarray*}
			{\footnotesize \textbf{mass terms:}}
			\begin{eqnarray*}
			\left( \overline{\hat{\psi^C}}_L \hat{\psi}_L + \overline{\hat{\psi}}_L \hat{\psi}^C_L \right)_{Majorana} \\
			 \left(\overline{\hat{\psi}}_L \hat{\psi}_R + \overline{\hat{\psi}}_R \hat{\psi}_L\right)_{Dirac}
			\end{eqnarray*}
			
		\end{columns}
	\end{frame}	
	
	\begin{frame}
		\frametitle{SM and BSM double-beta decay}
		\begin{columns}[c] % The "c" option specifies centered vertical alignment while the "t" option is used for top vertical alignment
			\column{.45\textwidth} % Left column and width
			{\footnotesize $\boldsymbol{2\nu\beta\beta}$: the neutrinos are Dirac fermions, and have distinct anti-particles $\nu$, $\overline{\nu}$} 
			\medskip
			\includegraphics[width=\textwidth]{~/research/figures/prelim/2vbb_feynman.png}
			{\footnotesize extremely rare process, observed in direct detection experiments (CUORE-0, GERDA, XENON, EXO, NEMO) $T_{1/2}^{2\nu\beta\beta} \sim 10^{19}-10^{24} yr$}
			
			\column{.5\textwidth} % Right column and width
			{\footnotesize $\boldsymbol{0\nu\beta\beta}$: the neutrinos are Majorana fermions and are absorbed as virtual particles}
			\includegraphics[width=\textwidth]{~/research/figures/prelim/0vbb_feynman.png}
			{\footnotesize Never before observed in any experiment:  $T_{1/2}^{0\nu\beta\beta} > 10^{26} yr$  (Lepton number violation)}
			
		\end{columns}
	\end{frame}		
	
	\begin{frame}
		\frametitle{experimental signature and detection}
		\begin{columns}[c] % The "c" option specifies centered vertical alignment while the "t" option is used for top vertical alignment
			
			\column{.45\textwidth} % Left column and width
			%\textbf{Heading}
			\begin{itemize}
				\setlength\itemsep{2em}
				\item mono-energetic peak with 2 electrons leaving the reaction.
				\item tomography of electron tracks can help reduced backgrounds, except for the $2\nu\beta\beta$ irreducible background
				\item experimental goal is to measure $T_{1/2}^{0\nu\beta\beta}$ to constrain the effective neutrino mass $m_{\beta\beta}$.
				\item this measurement complements measurements of $\Sigma$ and $m_{\beta}$ from cosmology and kinematic experiments, respectively.
			\end{itemize}
			
			\column{.5\textwidth} % Right column and width
			\includegraphics[width=5.7cm]{~/research/figures/prelim/0vbb_spectrum.png}
			\includegraphics[width=4cm]{~/research/figures/prelim/0vbb_tracks.png}
			
		\end{columns}
	\end{frame}			
	
	\begin{frame}
		\frametitle{neutrino mass and oscillations}
		\begin{columns}[c] % The "c" option specifies centered vertical alignment while the "t" option is used for top vertical alignment
			
			\column{.45\textwidth} % Left column and width
%			neutrinoless double beta decay has characteristic decay - can resolve in gaseous TPC the tracks. the only irreducible background is 2vbb.
			\begin{itemize}
				\setlength\itemsep{2em}
				\item Homestake Mine and the solar neutrino anomaly $^{37}Cl+\nu_e \rightarrow \ ^{37}Ar + e^-$ observes factor of 3 discrepancy in $\nu_e$ flux.
				\item Sudbury Neutrino Observatory (SNO) measures total neutrino flux: $\nu_{\mu}, \nu_{\tau}$ account for $\nu_e$ deficit.
				\item KAMLAND, a reactor neutrino experiment shows survival probability of electron neutrinos $P = 1 - sin^2 (2\theta)sin^2(\Delta m^2 L \slash 4E)$
				
			\end{itemize}
			
			\column{.5\textwidth} % Right column and width
			\includegraphics[width=\linewidth]{~/research/figures/prelim/survival_probability.png}
		
			
		\end{columns}
	\end{frame}
	
	\begin{frame}
		\frametitle{mass scale}
		\begin{columns}[c] % The "c" option specifies centered vertical alignment while the "t" option is used for top vertical alignment
			
			\column{.45\textwidth} % Left column and width
%			neutrinoless double beta decay has characteristic decay - can resolve in gaseous TPC the tracks. the only irreducible background is 2vbb.
			\begin{itemize}
				\setlength\itemsep{2em}
				\item Physicists  believed neutrinos to be massless for decades, due to Standard Model predictions (Higgs mechanism for leptons) and experimental data
				\item Neutrinos are orders of magnitude lighter than any other fundamental particle. This in addition to them being chargeless, makes it hard to measure their presence in experiments.
				\item The common model for describing BSM neutrino mass states is the light-neutrino mixing model (3 mass eigenstates)
				
				
			\end{itemize}
			
			\column{.5\textwidth} % Right column and width
			\includegraphics[width=\linewidth]{~/research/figures/prelim/mass_scale.png}
		
			
		\end{columns}
	\end{frame}
		
	\begin{frame}
		\frametitle{neutrino mixing}
			\begin{columns}
			\column{.45\textwidth}
			A given flavor eigenstate is a linear combination of mass eigenstates (PMNS-matrix)
			\begin{equation*}
			\begin{bmatrix}
			\nu_e \\
			\nu_{\mu} \\
			\nu_{\tau}
			\end{bmatrix}
			= U
			\begin{bmatrix}
			\nu_1 \\
			\nu_2 \\
			\nu_3
			\end{bmatrix}
			\end{equation*}			
			parameters: angles $\theta_{12},\theta_{13},\theta_{23}$ CP-violating phases $\delta_{CP}, \alpha_1, \alpha_2$, and neutrino masses $m_1, m_2, m_3$
			\column{.45\textwidth}			
			 
			\begin{equation*}		
			\resizebox{5.7cm}{!}{$U = \begin{pmatrix}
			c_{12}c_{13} & s_{12}c_{13} & s_{13}e^{-i\delta_{CP}} \\
			-s_{12}c_{23}-c_{12}s_{23}s_{13}e^{i\delta_{CP}} & c_{12}c_{23}-s_{12}s_{23}s_{13}e^{i\delta_{CP}}
			& s_{23}c_{13} \\ s_{12}s_{23}-c_{12}c_{23}s_{13}e^{i\delta_{CP}} & 
			-c_{12}s_{23}-s_{12}c_{23}s_{13}e^{i\delta_{CP}} & c_{23}c_{13}
			\end{pmatrix}$}
			\end{equation*}
			
			\begin{equation*}
			\resizebox{2cm}{!}{$
			\times 
			\begin{pmatrix}
			e^{i\alpha_1 /2} & 0 & 0 \\
			0 & e^{i\alpha_2 /2} & 0 \\
			0 & 0 & 1
			\end{pmatrix}$}
			\end{equation*}
			
			\end{columns}
	\end{frame}
	
	\begin{frame}
		\frametitle{mass hierarchy}
		\begin{columns}[c] % The "c" option specifies centered vertical alignment while the "t" option is used for top vertical alignment
			
			\column{.45\textwidth} % Left column and width
			Neutrino oscillation experiments are sensitive to $\Delta m_{ij}^2 = m_i^2 - m_j^2$ 
			This, coupled with the neutrinos' very small mass scale, leads to ambiguity in their ordering.
			\textbf{normal ordering} $\implies m_3^2 > m_2^2 > m_1^2$ \\
			\textbf{inverted ordering} $\implies m_2^2 > m_1^2 > m_3^2$
			\column{.5\textwidth} % Right column and width
			\includegraphics[width=6cm]{~/research/figures/prelim/hierarchy.png}

			
		\end{columns}
	\end{frame}
	
	%--------------------------------------------	
	
		
	
%	\begin{frame}
%		\frametitle{why look for this hypothetical process?}
%		\begin{columns}[c] % The "c" option specifies centered vertical alignment while the "t" option is used for top vertical alignment
%			\column{.45\textwidth} % Left column and width
%			\begin{itemize}
%				\setlength\itemsep{2em}
%				\item Historically, the weak interaction has been an active sector for studying and testing the standard model. C and P symmetry violation $\rightarrow$ CP symmetry as established symmetry. V-A form of the weak interaction.
%				\item therefore, for neutrinos to be massive, there must be some mechanism beyond the standard model that can give them mass.
%				\item lepton number conservation is an 'accidental symmetry' of the standard model
%				\item baryogenesis (matter / anti matter symmetry) can be explained by leptogenesis
%			\end{itemize}
%			
%			\column{.5\textwidth} % Right column and width
%			\includegraphics[width=\textwidth]{~/research/figures/prelim/0vbb_feynman.png}
%			
%		\end{columns}
%	\end{frame}		
	
	\begin{frame}
		\frametitle{effective Majorana neutrino mass}
		\begin{columns}[c] % The "c" option specifies centered vertical alignment while the "t" option is used for top vertical alignment
			
			\column{.45\textwidth} % Left column and width
			\begin{center}
			\includegraphics[width=6cm]{~/research/figures/prelim/mBB.png}
			\end{center}
			
			\column{.5\textwidth} % Right column and width
			\begin{center}
			\includegraphics[width=4cm]{~/research/figures/prelim/majoranamass.png}			
			\begin{equation*}
			\mathcircled{m_{\beta\beta}}\bar{\psi}\psi
			\end{equation*}
			\includegraphics[scale = 0.3]{~/research/figures/prelim/bsmparam.png}
			\end{center}
		\end{columns}
	\end{frame}
	
	\begin{frame}
		\frametitle{zero neutrino half-life}
			
		%\textbf{Heading}
		{\footnotesize 
		\begin{eqnarray*}
		T_{1/2}^{0\nu\beta\beta} &=\left(G|\mathcal{M}|^2 \langle m_{\beta\beta}\rangle\right)^{-1} \\
		\langle m_{\beta\beta}\rangle = \Sigma |U_{ej}^2 m_j|
		\end{eqnarray*}}
		\includegraphics[width=9cm]{~/research/figures/prelim/lobster.png}			
			
	\end{frame}	
	
	\begin{frame}
		\frametitle{complementary limits}
		\begin{columns}[c] % The "c" option specifies centered vertical alignment while the "t" option is used for top vertical alignment
			
			\column{.45\textwidth} % Left column and width
			%\textbf{Heading}
			\begin{itemize}
				\setlength\itemsep{2em}
				\item Limits on the effective $0\nu\beta\beta$ halflife are also constrained by kinematic and cosmological searches
				\item $m_{\beta} = \sqrt{U_{ei}^2 m_i^2}$ and $\Sigma = m_1 + m_2 + m_3$
				\item $m_{\beta} <$ 1.1eV \\ $\Sigma <$ 0.3eV
			\end{itemize}
			
			\column{.5\textwidth} % Right column and width
			\hspace*{-0.8cm}\includegraphics[width=7cm]{~/research/figures/prelim/mB_sigma.png}
			\footnotesize{\textit{\textcolor{red}{inverted} \textcolor{blue}{normal}}}
			
		\end{columns}
	\end{frame}
	
\begin{frame}
		\frametitle{far reaching consequences of observing $0\nu\beta\beta$}
		\begin{columns}[c] % The "c" option specifies centered vertical alignment while the "t" option is used for top vertical alignment
			
			\column{.45\textwidth} % Left column and width
			%\textbf{Heading}
			\begin{itemize}
				\setlength\itemsep{2em}
				\item baryogenesis through leptogenesis
				\item charge parity violating process in the standard model
				\item lepton number conservation (accidental symmetry of the standard model)
			\end{itemize}
			
			\column{.5\textwidth} % Right column and width
			\begin{gather*}
			m_{\nu} = U 
			\begin{pmatrix} m_1 & 0 & 0 \\
							0 & m_2 & 0 \\
							0 & 0 & m_3
			\end{pmatrix}
			U^{T}
			\end{gather*} 
			\begin{equation*}
			\langle m_{\beta\beta}\rangle = |U_{ij}^2 m_j|
			\end{equation*}
			
		\end{columns}
	\end{frame}	
	
	\begin{frame}
		\frametitle{$0\nu\beta\beta$ half-life and sensitivity}
		\begin{columns}[c] % The "c" option specifies centered vertical alignment while the "t" option is used for top vertical alignment
			
			\column{.45\textwidth} % Left column and width
			%\textbf{Heading}
			\begin{itemize}
				\setlength\itemsep{2em}
				\item direct detection experiments typically characterize the half-life in terms of the effective neutrino mass $m_{\beta\beta}$
			\end{itemize}
			
			\column{.5\textwidth} % Right column and width
			\begin{eqnarray*}
			T_{1/2}^{0\nu\beta\beta} &=\left(G|\mathcal{M}|^2 \langle m_{\beta\beta}\rangle\right)^{-1} \\
			& \simeq 10^{27-28} \left(\frac{0.01 \ eV}{\langle m_{\beta\beta}\rangle}\right)^2 \text{years}
			\end{eqnarray*}
			\begin{equation*}
			F_{0\nu} = \tau_{1/2}^{bkg. fluct.} = \text{ln}(2)N_{\beta\beta}\epsilon \frac{t}{n_B}
			\end{equation*}
		\end{columns}
	\end{frame}	
	
	\begin{frame}
		\frametitle{$0\nu\beta\beta$ half-life and sensitivity}
		\begin{columns}[c] % The "c" option specifies centered vertical alignment while the "t" option is used for top vertical alignment
			
			\column{.45\textwidth} % Left column and width
			%\textbf{Heading}
			\begin{itemize}
				\setlength\itemsep{2em}
				\item to probe
			\end{itemize}
			
			\column{.5\textwidth} % Right column and width
			\begin{eqnarray*}
			T_{1/2}^{0\nu\beta\beta} &=\left(G|\mathcal{M}|^2 \langle m_{\beta\beta}\rangle\right)^{-1} \\
			& \simeq 10^{27-28} \left(\frac{0.01 \ eV}{\langle m_{\beta\beta}\rangle}\right)^2 \text{years}
			\end{eqnarray*}
			\begin{equation*}
			T_{1/2}^{0\nu\beta\beta} \sim \  \epsilon \sqrt{\frac{M_{iso} t}{b \Delta E}}
			\end{equation*}
			
		\end{columns}
	\end{frame}	
	
	
	
\end{document}