
% ****** Start of file apssamp.tex ******
%
%   This file is part of the APS files in the REVTeX 4.1 distribution.
%   Version 4.1r of REVTeX, August 2010
%
%   Copyright (c) 2009, 2010 The American Physical Society.
%
%   See the REVTeX 4 README file for restrictions and more information.
%
% TeX'ing this file requires that you have AMS-LaTeX 2.0 installed
% as well as the rest of the prerequisites for REVTeX 4.1
%
% See the REVTeX 4 README file
% It also requires running BibTeX. The commands are as follows:
%
%  1)  latex apssamp.tex
%  2)  bibtex apssamp
%  3)  latex apssamp.tex
%  4)  latex apssamp.tex
%
\documentclass[%
 reprint,
%superscriptaddress,
%groupedaddress,
%unsortedaddress,
%runinaddress,
%frontmatterverbose, 
%preprint,
showpacs,
%preprintnumbers,
%nofootinbib,
%nobibnotes,
%bibnotes,
 amsmath,amssymb,
 aps,
%pra,
%prb,
%rmp,
%prstab,
%prstper,
%floatfix,
longbibliography,
]{revtex4-1}

\usepackage{graphicx}% Include figure files
\usepackage{dcolumn}% Align table columns on decimal point
\usepackage{multirow}
\usepackage{bm}% bold math
\usepackage{hyperref}% add hypertext capabilities
\usepackage[mathlines]{lineno}% Enable numbering of text and display math
%\linenumbers\relax % Commence numbering lines

%\usepackage[showframe,%Uncomment any one of the following lines to test 
%%scale=0.7, marginratio={1:1, 2:3}, ignoreall,% default settings
%%text={7in,10in},centering,
%%margin=1.5in,
%%total={6.5in,8.75in}, top=1.2in, left=0.9in, includefoot,
%%height=10in,a5paper,hmargin={3cm,0.8in},
%]{geometry}

\begin{document}

\preprint{qoqp}

\title{Superconducting Qubits: Progress Report 4}% Force line breaks with \\
%\thanks{A footnote to the article title}%

% All people from 1st institution
\author{Joseph Camilleri}
%\homepage{http://www.first.institution.edu/~Charlie.Author}
%\altaffiliation[Also at ]{Physics Department, XYZ University.}%Lines break automatically or can be forced with \\
\affiliation{Physics Department, Virginia Tech}%
%\collaboration{SuperCDMS}%\noaffiliation

%
%\collaboration{SuperCDMS}%\noaffiliation

\date{10/5/2020}% It is always \today, today,
             %  but any date may be explicitly specified

\begin{abstract}


This is the second half of chapter 2 of Bishop's thesis on cQED for transmon qubits
Revisit the notes or ask Sophia since she did this lecture, how does this Hamiltonian relate to tunneling of cooper pair bound states across the insulating junction?
\begin{equation*}
H = 4E_c \Sigma_{j=-N}^{j=N} \left(j-n_g\right)^2 |j> <j| - \dfrac{E_j}{2} \Sigma_{j=-N}^{j=N-1} \left(|j+1> <j| + |j> <j+1|\right)
\end{equation*}

I have some nomenclature confusion. The Cooper-pair number operator is *not* a number operator in the sense of raising and lowering operators (typically, n = adagger a). rather, it's like a position/momentum canonical conjugate variable that can be expressed in terms of raising and lowering operators. These raising and lowering operators, b and bdagger, raise lower the energy eigenstates of our duffing oscillator.

\begin{eqnarray*}
H = 4E_c n^2 - E_J + \dfrac{E_J\varphi^2}{2} - \dfrac{E_J\varphi^4}{24} \langle dude \rangle
\end{eqnarray*}
Flux tuning and the split transmon. Two Josephson junctons in parallel have a Hamiltonian as follows

\begin{equation*}
H = 
\end{equation*}


This is the section on dielectric loss modeled as a continuum statistical fermionic bath of TLS.

For cavity-qubit coupled systems, we generally consider loss due to a bosonic bath (e.g. photons and bosons / light and vibration).
Dissipation in dielectrics presents a new dissipation mechanism - an equivalent loss to a fermionic bath.
This can be understood physically as a power dissipated in a two level system (TLS). These TLS losses occur in two different dielectric boundaries in superconducting micro resonators: the insulating junction for the Josephson junction and the interface between the superconducting film / crystalline substrate interface.

how does the loss tangent relate to the quality factor for a generic resonator? Q is the narrowness of the pass band. This is also all related to decoherence times.

Losses to the fermionic bath are quantified by the \textit{loss tangent}

\begin{equation*}
tan(\delta) = \dfrac{Im(\epsilon)}{Re(\epsilon)}
\end{equation*}

where a smaller value indicates a smaller imaginary dielectric component, and hence less loss (wave vector \textit{k} will also be complex).

For a perfect supercodnductor / crystal interface the dielectric losses at low temperatures are totally negligible.
However, in practice, and further insulating layer of amorphous silicon, SiO2, is necessary for crossover wiring needed in circuit design (what is crossover wiring?).
What other reasons are there for amorphous silicon in microfabrication?

At low temperatures and driving frequencies $k_B T \ll \hbar \omega $

The dielectric loss exhibits behavior inconsistent with a bosonic bath model: it varies strongly with driving amplitude and scales inversely with the resonator power in frequency space 

\begin{equation*}
\tan(\delta) \simeq \dfrac{1}{\sqrt{<V^2 >}}
\end{equation*}

until it reaches its maximum value at small amplitudes. 

Explain why a resistor is modelled as a bosonic bath. Macroscopically, resistors are linear with voltage, so this non-linear behavior doesn't agree with a resistor model. something with random walks? maybe ask tauber about this, or does it come in from stat mech II? what does resistance have to do with 'transport properties'?

Additionally, at higher temperatures we observe LESS loss in the dielectric (this is weird, it must have something to do with fermi-dirac statistics).

The physical mechanism resposible for this loss: the resonant absorption of radiation by a bath of TLS possessing electric dipole moments.





\begin{eqnarray*}
\delta = \dfrac{\pi\rho (ed)^2}{3\epsilon}\dfrac{tanh(\dfrac{\hbar\omega}{2k_B T})}{1+\omega_{Rabi}T_1 T_2} \\ _
\hbar\omega_Rabi = \dfrac{(eVd)}{x}
\end{eqnarray*}

\textbf{this is from page 13 of zmuidzinas}
\textit{TLS arise due to the random structure of amorphous materials, because occa-
sionally it is possible for an atom or group of atoms to move between two local minima of the
potential energy landscape by quantum tunneling over a barrier. The random nature of the
amorphous material implies that the potential energy minima and the barrier height are also
random, leading to a random, uniform distribution of TLS energy splittings. In general, the TLS
are electrically active because the moving atoms carry a dipole moment, and therefore the TLS
make a contribution e TLS (o, T) to both the real (reactive) and imaginary (dissipative) parts of
the dielectric constant e. In weak fields the loss tangent due to TLS is given by
}

The microwave loss is dominated by TLS with energies ħo; the familiar hyperbolic tangent
factor arises from the thermal occupation probabilities of the two quantum states. For k B T <<
ħo, the TLS only occupy the ground state, so d TLS (o, T ) ! d 0 . The value of d 0 is proportional to
the density of TLS per unit volume and energy. The earliest models (68, 69) suggested that the
low-temperature loss tangent is independent of frequency because the TLS are uniformly
distributed in energy; later studies (70) indicate that the distribution function is slowly increas-
ing with energy in the RF and microwave range. In the opposite limit k B T << ħo, the upper
state is well populated and stimulated emission cancels much of the absorption, leading to
d TLS (o,T) /ħo/k B T.
The corresponding behavior of Re e TLS (o, T) can be found by applying the Kramers-Kronig
transform, and causes the resonator frequency to shift by an amount (71, 72)

\end{abstract}

%\pacs{29.40.Wk, 95.55.Vj, 95.35.+d}% PACS, the Physics and Astronomy
                             				% Classification Scheme.
%\keywords{Suggested keywords}%Use showkeys class option if keyword
                              %display desired
\maketitle

%\tableofcontents



%\bibliographystyle{IEEEtranN}
\bibliographystyle{aipnum4-1}
%\bibliographystyle{unsrt}
\bibliography{Bib_SCQB}

\end{document}


